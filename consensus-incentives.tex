\documentclass[11pt,a4paper]{article}

\setlength {\marginparwidth }{2cm}

\usepackage[margin=2.5cm]{geometry}
\usepackage{todonotes}
\usepackage[automake]{glossaries}
\usepackage{graphicx}
\usepackage{microtype}
\usepackage{listings}
\usepackage{color}
\usepackage{amssymb,amsmath}
\usepackage{mathpazo}
\usepackage{accents}
\usepackage{longtable,booktabs}
\usepackage{dcolumn}
\usepackage{pgf}
\usepackage{mathrsfs}
\usetikzlibrary{shapes.geometric}
\usetikzlibrary{matrix}
\usepackage{natbib}
\usepackage{hyperref}
\usepackage[capitalise,noabbrev,nameinlink]{cleveref}
\hypersetup{
  pdftitle={Algorand Consensus Incentivisation},
  pdfborder={0 0 0},
  breaklinks=true
}
\graphicspath{ {./images/} }

\definecolor{dkgreen}{rgb}{0,0.6,0}
\definecolor{gray}{rgb}{0.5,0.5,0.5}
\definecolor{pink}{rgb}{0.92,0.2,0.86}

\lstset{frame=tb,
  language=C,
  aboveskip=3mm,
  belowskip=3mm,
  showstringspaces=false,
  columns=flexible,
  basicstyle={\ttfamily},
  numbers=none,
  numberstyle=\tiny\color{pink},
  keywordstyle=\color{pink},
  commentstyle=\color{pink},
  stringstyle=\color{pink},
  breaklines=true,
  breakatwhitespace=true,
  tabsize=3
}

\DeclareMathOperator{\dom}{dom}
\newcommand\restrict[2]{\left.#1\right||_{#2}}
\newcommand\deltavar[1]{\accentset{\Delta}{#1}}

\makeglossaries
\newglossaryentry{PPOS}
{
    name={Pure Proof of Stake},
    description={a consensus algorithm based on verifiable random functions over elliptic curves, which provide a 
                 mechanism of cryptographic sortition. Naturally incentivises non-pooled block production.}
}
\newglossaryentry{POW}
{
    name={Proof of Work},
    description={a consensus algorithm based on symmetric hash functions, where a newly mined block's digest is under a 
                 specific threshold. Does not naturally incentivise non-pooled block production.}
}
\newglossaryentry{Staker}
{
    name={Staker},
    description={An individual who is actively participating in the security of the network by running a participation 
                node, with valid participation keys, backed by a sum of Algo}
}
\newglossaryentry{Block-Reward}
{
    name={Block-Reward},
    description={a sum of Algo paid to the participation account of an active Staker, when the Staker has successfully
                produced a valid and network-wide accepted block.}
}

\begin{document}

\title {Algorand Consensus Incentivisation \\
       {\large \sc An Algorand Foundation discussion paper}}
\date  {Version 0.2, 16th August 2023}
\author{
    John Woods \\ 
    {\small \texttt{john@algorand.foundation}} \\
    {\small \texttt{john@postquantum.dev}} \\
\and 
    Michele Treccani  \\
    {\small \texttt{michele@algorand.foundation}}
}

\maketitle

\section{Purpose}
The goal of this project is to engineer a native solution at layer 1, modifying the Algorand protocol to 
\emph{incentivise} the execution of consensus via block production rewards in ALGO. \\

\textbf{The technical goals of this project are:}

\begin{enumerate}
    \item Define the economic design (source of funds, payout function, non-functional requirements) - this is the 
          primary Foundation output.
    \item Define the technical design (protocol level augmentations, code-level changes, functional requirements) 
          - secondary goal, Inc will lead.
    \item Define the approach to mitigate gamification (Key surrender, Absenteeism, Vote neglect) - secondary goal.
\end{enumerate}

\pagebreak

\tableofcontents

\pagebreak

\section{Acknowledgements}
The ideas presented herein are inspired by and based on discussions with individuals from Algorand Foundation, 
Algorand Inc, including the research and engineering groups, specifically: Michele Treccani, Silvio Micali, Paul Riegle, 
Gary Malouf, Eric Wragge, Staci Warden \& Bruno Martins.

\pagebreak

\section{Context}
Algorand employs \gls{PPOS} (glossary below) as a means of reaching consensus. The security of the decentralised network 
is predicated on the total count of Algo tokens which are actively staked at any given moment, as a percentage of the 
total circulating supply. Notably, Algorand's security assumptions rely upon at least $\sim$67\% of the underlying stake 
being honest in order for the network to remain impervious to attack, rather than $\sim$51\% in traditional \gls{POW} 
based systems. \\ 

Algorand's original thesis contended, a priori, that a sufficient number of end-users would be naturally incentivised to 
run a node and contribute to active stake, motivated by the desire to secure their own funds, and as a corollary the 
network. However, a posteriori, it has been observed that as a consequence of the cost to run a participation node with 
high uptime, and indeed the technical expertise required to both instantiate and maintain a node, a critical mass of 
end-user contribution has not been reached. \\

In fact, as the distribution of large holders has slowly fragmented, and a larger proportion of the supply has been held 
by an increasingly less concentrated set of individuals, active stake has notably \emph{dropped}. The aforementioned 
friction is too much for most individuals to contribute actively to network security. \\ \\

This paper proposes potential solutions, at a protocol level, to encourage end-users to actively stake their Algo.
Additionally, we introduce the concept of financial incentives to reward staking, with the aim of materially increasing 
the total percentage of Algo in circulating supply which are participating in consensus. 

\pagebreak

\section{Solution design}

\subsection{Economic considerations}

\subsection{Payout function}
For the purpose of discussion, defined below is a linear reward function, which at a given point in time (defined 
by round number $\eta$) evaluates to a \gls{Block-Reward} in Algo:\\

\textbf{First we define the emission shape, with a linear decrease:}
\[
R(\eta) = (1- \frac{\eta}{\mathcal{N}})
\]

\textbf{then, adding a normalisation factor:}
\[
\mathcal{K} = \sum_{\eta=1}^\mathcal{N}R(\eta)=(\mathcal{N} -  \frac{\mathcal{N}(\mathcal{N}+1)}{2\times \mathcal{N}})  
\]

\textbf{$\ni$ the payout function is defined as:}
\[
\mathcal{R}(\eta) = \mathcal{M} \times R(\eta) \times \frac{1}{\mathcal{K}}
\]

\textbf{\emph{Where:}}
\begin{align*}
\mathcal{M} & : \text{Total coins for incentive (e.g., 500 million).} \\
\mathcal{B} & : \text{Block time (e.g., 3 seconds).} \\
\mathcal{T} & : \text{Total duration for the payout (e.g., 5 years).} \\
\mathcal{N} & : \text{Total number of blocks over the payout period} = \frac{\mathcal{T}}{\mathcal{B}}.
\end{align*}

Whilst a linear payout is reasonable, we could also consider a function with a configurable exponential decay, which may 
be more fitting given Algorand's capped, inherently deflationary supply: \\ 

\textbf{First we define the emission shape, with an exponential decrease:}
\[
R'(\eta)= e^{-\rho\frac{\eta}{\mathcal{N}}}
\]

\textbf{then, adding a normalisation factor:}
\[
\mathcal{K} = \sum_{\eta=1}^\mathcal{N}R'(\eta)= \frac{1-e^{-\rho}}{1-e^{-\frac{\rho}{\mathcal{N}}}}
\]

\textbf{$\ni$ the payout function is defined as:}
\[
\mathcal{R'}(\eta)= \mathcal{M}\times e^{-\rho\frac{\eta}{\mathcal{N}}} \frac{1}{\mathcal{K}}
\]

Here the shape of the curve would depend on the value of \(\mathcal{K}\). Where the lower the value of \(\mathcal{K}\), 
the more elongated the curve.

\subsection{Frequency}
The payout function for the \glsplural{Block-Reward} will execute on a periodic basis. The payout will occur every 
$\eta$ blocks. At the time of writing the average block time is \emph{3.4 seconds}.

\subsection{Game mitigations}

\subsection{Attacks/Fragility}
\todo{lorem}

\subsection{Fees}

\pagebreak

\section{Conclusion}

\pagebreak

\printglossaries

\end{document}
