\documentclass[11pt,a4paper]{article}

\setlength {\marginparwidth }{2cm}

\usepackage[margin=2.5cm]{geometry}
\usepackage{todonotes}
\usepackage[automake]{glossaries}
\usepackage{graphicx}
\usepackage{microtype}
\usepackage{listings}
\usepackage{color}
\usepackage{amssymb,amsmath}
\usepackage{mathpazo}
\usepackage{accents}
\usepackage{longtable,booktabs}
\usepackage{dcolumn}
\usepackage{pgf}
\usepackage{mathrsfs}
\usetikzlibrary{shapes.geometric}
\usetikzlibrary{matrix}
\usepackage{natbib}
\usepackage{hyperref}
\usepackage[capitalise,noabbrev,nameinlink]{cleveref}
\hypersetup{
  pdftitle={Algorand Consensus Incentivisation},
  pdfborder={0 0 0},
  breaklinks=true
}
\graphicspath{ {./images/} }

\definecolor{dkgreen}{rgb}{0,0.6,0}
\definecolor{gray}{rgb}{0.5,0.5,0.5}
\definecolor{pink}{rgb}{0.92,0.2,0.86}

\lstset{frame=tb,
  language=C,
  aboveskip=3mm,
  belowskip=3mm,
  showstringspaces=false,
  columns=flexible,
  basicstyle={\ttfamily},
  numbers=none,
  numberstyle=\tiny\color{pink},
  keywordstyle=\color{pink},
  commentstyle=\color{pink},
  stringstyle=\color{pink},
  breaklines=true,
  breakatwhitespace=true,
  tabsize=3
}

\DeclareMathOperator{\dom}{dom}
\newcommand\restrict[2]{\left.#1\right||_{#2}}
\newcommand\deltavar[1]{\accentset{\Delta}{#1}}


\makeglossaries
\newglossaryentry{PPOS}
{
    name={Pure Proof of Stake},
    description={a consensus algorithm based on verifiable random functions over elliptic curves, which provides a
                 mechanism of cryptographic sortition. Naturally incentivises non-pooled block production.}
}
\newglossaryentry{POW}
{
    name={Proof of Work},
    description={a consensus algorithm based on symmetric hash functions, where a newly mined block's digest is under a
                 specific threshold. Does not naturally incentivise non-pooled block production.}
}
\newglossaryentry{Staker}
{
    name={Staker},
    description={an individual who is actively participating in the security of the network by running a participation
                 node, with valid participation keys, backed by some sum of Algo.}
}
\newglossaryentry{Block-Reward}
{
    name={Block-Reward},
    description={a sum of Algo paid to the participation account of an active Staker, when the Staker has successfully
                 produced a valid and network-wide accepted block.}
}
\newglossaryentry{FeeSink}
{
    name={Fee Sink},
    description={an account, (Y76M3MSY6DKBRHBL7C3NNDXGS5IIMQVQVUAB6MP4XEMMGVF2QWNPL226CA), which receives all Fees 
                 generated by transactions on Algorand mainnet.}
}
\newglossaryentry{KeyReg}
{
    name={Key Registration Transaction},
    description={a native Algorand transaction type that is used to register an underlying account's balance as an 
                 active part of the global stake utilised by Consensus.}
}
\newglossaryentry{SpendKey}
{
    name={Spending Key},
    description={the primary private key for a given Algorand account, confers the right to spend/move the account's 
                 funds upon the holder.}
}
\newglossaryentry{PartKey}
{
    name={Participation Key},
    description={a secondary private key, associated with the primary private key/account via signature from the primary
                 private key, imbued with the right to sign messages on behalf of the associated account during 
                 consensus.}
}
\newglossaryentry{Blockseed}
{
    name={Blockseed},
    description={a pseudo-random number generated for each block. Its value is based on various pieces of information 
                 from the blockchain. It is used as part of the sortition algorithm.}
}

\begin{document}

\title {Algorand Consensus Incentivisation \\
       {\large \sc An Algorand Foundation discussion paper}}
\date  {Version 0.9, 13th December 2023}
\author{
    John Woods \\ 
    {\small \texttt{john@algorand.foundation}} \\
    {\small \texttt{john@postquantum.dev}} \\
\and 
    Michele Treccani \\
    {\small \texttt{michele@algorand.foundation}}
\and 
    John Jannotti \\
    {\small \texttt{john.jannotti@algorand.com}}
\and 
    Naveed Ihsanullah \\
    {\small \texttt{naveed@algorand.foundation}}
}

\maketitle

\section{Purpose}
The goal of this project is to engineer a native solution at layer 1, modifying the Algorand protocol to 
\emph{incentivise} the execution of consensus via block production rewards in Algo, Algorand's native token. \\

\textbf{The technical goals of this project are:}

\begin{enumerate}
    \item Define the economic aspects of the design (source of funds, payout function, non-functional requirements). 
        
    \item Define the technical design (protocol level augmentations, code-level changes, functional requirements). 
        
    \item Define the approach to mitigate gamification ("Key surrender", "Absenteeism", "Vote neglect").
\end{enumerate}

\pagebreak

\tableofcontents

\pagebreak

\section{Acknowledgements}
The ideas presented herein are inspired by and based on discussions with individuals from Algorand Foundation \& 
Algorand Technologies (previously Inc), including the research and engineering groups, specifically: Silvio Micali, 
Paul Riegle, Gary Malouf, Eric Wragge, \& Bruno Martins.

\pagebreak

\section{Context}
Algorand employs \gls{PPOS} (glossary below) as a means of reaching consensus. The security of the decentralised network 
is predicated on the total count of Algo tokens which are actively staked at any given moment, as a percentage of the 
total circulating supply. Notably, Algorand's liveness relies upon at least $\sim$80\% of the participating ("online") 
stake to actively and honestly taking part in the consensus protocol in order to produce blocks. \\ 

Algorand's original thesis contended, a priori, that a sufficient number of end-users would naturally be incentivised to 
run a node and contribute to online stake, motivated by the desire to secure their own funds, and as a corollary the 
network. However, a posteriori, it has been observed that as a consequence of the cost to run a participation node with 
high uptime, and indeed the technical expertise required to both instantiate and maintain a node, a critical mass of 
end-user contribution has not been reached. \\

In fact, as the distribution of Algo across accounts has become more uniform, and a larger proportion of the supply has 
become held by an increasingly less concentrated set of individuals, online stake has notably \emph{dropped}. The 
aforementioned friction is seemingly too much for most individuals to contribute actively to network security. \\ \\

This paper proposes a potential solution, at a protocol level, to encourage end-users to actively participate with their 
Algo. Additionally, we introduce the concept of financial incentives to reward participation, with the aim of materially 
increasing the total percentage of Algo in circulating supply which is participating in consensus. 

\pagebreak

\section{Solution design}
Here we outline the technical solution design for payment of consensus incentives, dynamically by the protocol, without 
manual intervention or processes. 

\subsection{Incentives from Fees}
Henceforth, we use the term “mined incentives” to refer to the proportion of incentives yielded from the act of 
proposing a block which are directly attributable to Fees. As of 2023, the global Fee volume is unsubstantive. That 
said, we should introduce this mechanism now as any sustainable approach to consensus incentivsation will necessitate 
the consumption of Fees to pay block proposers. \\

The mining mechanism must be built into the L1 protocol, as it diverts funds from the global \gls{FeeSink}. The 
following technical approach should be sufficient to correctly pay out mined incentives.

\begin{enumerate}
    \item When \texttt{proto.EnableMining}, add \texttt{Proposer} and \texttt{FeesCollected} fields to the block header.
    \item Determine \texttt{proto.MiningPercent} which is the percentage of Algo paid in Fees that should go to the 
          proposer of the block instead of the \gls{FeeSink}.
    \item At the start of block \texttt{n+1}, move \texttt{MiningPercent * block[n].FeesCollected} from the 
          \gls{FeeSink} to \texttt{block[n].Proposer}.
    \item Outlaw spending from the \gls{FeeSink} to ensure the funds above are available in block \texttt{n+1}.
\end{enumerate}

The movement is executed at the start of a given block \texttt{n+1} instead of during block \texttt{n} because a 
multi-participant node builds a block \textit{before} it knows which of its accounts might actually be the proposer. 
Between blocks, the funds remain in the \gls{FeeSink}.

\subsection{Economic considerations} \label{subsec:eco}
In addition to the incentives generated by the Fee market, in the short to medium term additional funds will be required
to ensure incentives are substantive whilst the network's transaction volume increases. Thus the Algorand Foundation 
will contribute an additional sum of Algo to supplement the incentives generated by Fees. \\ \\
To the end user, the \gls{Block-Reward} paid to the proposers account will be congruent, that is, the portion of the
reward which represents the AF contribution, and the portion of the reward which represents the Fee contribution will
seamlessly combine.

\pagebreak

\subsection{Payout function}
The combined incentives (Fees and supplemental incentives) described above will payout following a time dependent 
function, where the emission implies deflation. This function will of course be unnecessary when supplemental incentives
are exhausted, at which point Fees scaled by \texttt{proto.MiningPercent} will become the sole payout per block.

Consider the following two emission function definitions:

Defined below is a linear reward function, which at a given point in time (defined by round number $\eta$) evaluates to 
a \gls{Block-Reward} in Algo:\\

\textbf{First we define the emission shape, with a linear decrease:}
\[
R(\eta) = (1- \frac{\eta}{\mathcal{N}})
\]

\textbf{then, adding a normalisation factor:}
\[
\mathcal{K} = \sum_{\eta=1}^\mathcal{N}R(\eta)=(\mathcal{N} -  \frac{\mathcal{N}(\mathcal{N}+1)}{2\times \mathcal{N}})  
\]

\textbf{$\ni$ the payout function is defined as:}
\[
\mathcal{R}(\eta) = \mathcal{M} \times R(\eta) \times \frac{1}{\mathcal{K}}
\]

\textbf{\emph{Where:}}
\begin{align*}
    \mathcal{M} & : \text{Total units for incentive (e.g., 500 million Algo).} \\
    \mathcal{B} & : \text{Block-time (e.g., 3 seconds).} \\
    \mathcal{T} & : \text{Total duration for the payout, expressed in seconds (e.g., 5 years).} \\
    \mathcal{N} & : \text{Total number of blocks over the payout period} = \frac{\mathcal{T}}{\mathcal{B}}.\\
\end{align*}

Whilst a linear payout is \textit{reasonable}, we could also consider a function with a configurable exponential decay 
(also defined by round number $\eta$), which may be more fitting given Algorand's existing capped, inherently 
deflationary supply: \\ 

\textbf{First we define the emission shape, with an exponential decrease:}
\[
R'(\eta)= e^{-\rho\frac{\eta}{\mathcal{N}}}
\]

\textbf{then, adding a normalisation factor:}
\[
\mathcal{K} = \sum_{\eta=1}^\mathcal{N}R'(\eta)= \frac{1-e^{-\rho}}{1-e^{-\frac{\rho}{\mathcal{N}}}}
\]

\textbf{$\ni$ the payout function is defined as:}
\[
\mathcal{R'}(\eta)= \mathcal{M}\times e^{-\rho\frac{\eta}{\mathcal{N}}} \frac{1}{\mathcal{K}}
\]

\textbf{\emph{Where:}}
\begin{align*}
\rho & : \text{Rate of decay.} \\
\end{align*}

Here the shape of the curve would depend on the value of \(\mathcal{K}\). Where the lower the value of \(\mathcal{K}\), 
the more elongated the curve. \\

\emph{Note: With the development of "Dynamic $\lambda$" (essentially dynamic round-times, where the time-to-wait for the
lowest VRF proposal is dynamically calculated, and optimised to ignore some slowest percentile of proposers), it is 
likely prudent to consider a mechanism to temporally tweak the payout functions proposed above, as with dynamic 
round-times, $\eta$ will no longer provide a linear approximation of time. As such, we may consider assuming an 
aggregate block-time for $\mathcal{B}$. Such an assumption can then be adjusted if significant deviation from our 
expected $\mathcal{T}$ occurs.}

\subsection{Frequency}
The payout function for the \glsplural{Block-Reward} will execute on a periodic basis. The payout will occur every 
$\eta$ block(s). At the time of writing the average block time is \emph{3.3 seconds}, and the expected payout is 
\textit{per round/block}. 

\pagebreak

\subsection{Game mitigations/Anti-fragility}
Below we describe the primary attack vectors and game-theory aspects as they pertain to the proposed implementation 
above.

\subsubsection{Absenteeism} \label{subsubsec:absenteeism}

\paragraph{Concern:} \mbox{}\\
Absenteeism refers to stakeholders registering their Algorand account online, but subsequently failing to participate in 
protocol operations (such as proposing and validating blocks). If a sufficient percentage of active stake declares 
itself online but does not do the work to participate in the consensus protocol the blockchain will stall. Recovery from 
this stall requires that the derelict stake eventually participates or a network hard-fork is executed to bypass the 
defaulting Algo from being considered. 

\paragraph{Mitigation:} \mbox{}\\
We propose a novel solution to Absenteeism which essentially empowers the network to suspend delinquent participants/
accounts from the staking set. Where that participant/account has failed to execute their incumbent responsibilities.
As part of this solution we will introduce a new transaction type which will supplement the standard \gls{KeyReg} 
transaction mechanism with a new type dedicated to "re-enrollment" of an account in the staking set should it have been 
suspended from said staking set by the network mistakenly.

This will ensure false positives have a mechanism by which they can re-engage in consensus without downtime.
\emph{Note: this solution operates at a consensus level.}

The approach is as follows:

\begin{itemize}
    \item Consider one's stake always represents some $\mathcal{X}\%$ of $\mathcal{Y}$, one's proportion of the global 
          stake.
    \item One's stake will imply a probability distribution across the staking set, defining a threshold with regard to 
          the probable number of successful block proposals per account, a quantitative threshold which can be used to 
          assess the likelihood of an individual acting in the best interest of the network.
    \item If an account falls foul of this threshold, the network will suspend the participant from the staking
          set, for example where an account's count of successful proposals is significantly below the statistically 
          implied threshold.
    \item There is of course a possibility that an individual was just 'unlucky' rather than inactive. In such 
          circumstances their node can trivially post a re-enrollment transaction which is, critically, signed by the 
          node's \gls{PartKey}, not their \gls{SpendKey}.
    \item Thus, the network can suspend misbehaving entities, including those who may not be bad actors but whose stake 
          is unreliable and detrimental to the network, whilst simultaneously providing a mechanism for entities who
          were incorrectly suspended to reassert their commitment to the network and continue to earn rewards.
    \item To avoid suspension in cases where an account may simply receive a large portion of Algo which skews their 
          implied threshold, a \emph{constant} grace period will be observed prior to suspension to allow accounts to 
          proactively respond, prior to any action being applied.
    \item This approach will utilise \glsplural{PartKey}, and not \glsplural{SpendKey}. The process will
          also not require additional \gls{KeyReg} transactions.
    \item This approach effectively provides "Garbage Collection" for consensus, permitting relatively significant 
          portions of global stake (5\% to 10\%), within a reasonable \emph{rolling} time period, to ungracefully exit 
          consensus ad-infinitum without detriment to the network.
    \item As an additional check of liveness, nodes could be requested to ack in response to challenges based 
          on \glsplural{Blockseed}, (which are notably unpredictable and impossible to influence). This approach would 
          permit a targetted percentage subset of the active staking set to be "assessed", asserting their liveness, and
          suspended where an ack is not provided. This mechanism could be executed every \(\mathcal{Z}\) rounds, with 
          the set of participants based on some deterministic slice of the \gls{Blockseed}, for example by comparing 
          each staker's address with some \(\mathcal{B}\) byte portion of the \gls{Blockseed}. The challenge would also 
          be predicated on \gls{Blockseed}, and be trivially computable.
    \item In order to discourage unreliable consensus execution, when an account is suspended it should incur a small
          but substantive expense in order to return to online status. We may want to bound this cost with an 
          integer limit value in the initial \gls{KeyReg}, controlling the number of times an account can reassert. This
          bound allows a security conscious participant to be sure that their \gls{PartKey} cannot be used to "drain"
          their account.

\end{itemize} \mbox{} \\

Finally, minimum and maximum staking notionals may be enforced in order for an account to receive a \gls{Block-Reward}.

A minimum, for example, $8,192$ Algo, $16,384$ Algo, $32,768$ Algo, would serve two purposes, first, it would set a hard
upper bound on the total number of incentive-eligible participating nodes. This bound makes it practical to track all 
participating accounts in active memory, which is required to ensure absenteeism checks are time bound, second it would 
likely deter intermittent node operations by "nano" accounts. \\ 

A staking ceiling encourages large accounts to split their stake across multiple accounts, which, where split across
multiple nodes, adds stall resilience to the network's consensus. Such a limit would likely be between $2^{16}$ and
$2^{17}$ Algo. \\ \\

\pagebreak

\subsubsection{Pooling}

\paragraph{Concern:} \mbox{}\\
The liveness of the Algorand protocol requires that $\sim$80\% of the participating stake is held by honest 
participants. While participation in Algorand's consensus algorithm is neither computationally burdensome nor 
operationally complex in an absolute sense, it may still prove to be beyond the technical capability of the average 
Algorand user, who may not be familiar with managing computer systems. These participants may be likely to pool their 
Algo with a node operator who promises to participate on their behalf and return their rewards for an operational fee. 
Today this is possible without risking the Algo owner’s principal because participation activity involves a secondary 
\gls{PartKey}, associated with, but not controlling spend from the payment account. If users choose to delegate their 
\gls{PartKey} to a pooling service, this may concentrate stake with a single entity, posing a risk to network safety 
through potential accidents or malicious actions, where, as outlined in the section above on Absenteeism, 
\ref{subsubsec:absenteeism} a stall could occur. \\ \\

This scenario cannot be meaningfully mitigated by limiting the maximum participation from a particular account, 
node or IP address as a Sybil-attack will trivially bypass such protections.

\paragraph{Mitigation:} \mbox{}\\
Some high-level mitigations and anti-fragility measures have already been baked into the protocol. For example, if a
\gls{PartKey} expires without re-registration, the network will gracefully remove the participant from the staking set.
Additionally, the Foundation will work directly with potential pool operators to provide education and set operational
best practice, including providing play-books. \\ \\
Properly run pools may be beneficial to the network, where appropriate volumes of stake are separated over
participation nodes which are professionally run and exhibit high-availability. Finally, an over-abundance of online 
accounts on a given participation node is naturally discouraged, running a participation node with more than $/sim$3 
accounts present will result in a degradated performance during execution of consensus, notably resulting in lost 
rewards.

\pagebreak

\subsubsection{Modified Clients}

\paragraph{Concern:} \mbox{}\\
A strategy to earn consensus rewards whilst also minimizing the network and computational resources required whilst 
participating, is to participate through a bespoke, modified, Algorand node. The official Algorand node implementation 
maintains substantial local state reflecting the current balance tree of the Algorand blockchain. This reference 
implementation is usually executed on medium-end hardware to ensure that the responsibility of consensus execution (such 
as block proposal from current queued transactions and proposal verification based on current balances) can be completed 
in the smallest quanta of time. Network and computational resources may also be required for the message passing and  
algorithmic processing required. Thus, a sophisticated attacker may consider building a customised node implementation 
which avoids the primary resource intensive activities such as keeping in parity with other node's blockchains, 
proposing empty blocks or falsely validating proposed blocks without actually checking the payload - or equivalent 
shortcuts. 

\paragraph{Mitigation:} \mbox{}\\
Whilst we thought it prudent to consider the potential advantages from such an approach, there does not appear to be a 
meaningful benefit with this strategy. The computational burden from execution of the VRF and validation of contracts 
is negligible. The benefit derived from developing and executing such a customised node is not commensurate with the 
effort required to undertake such a strategy, in a \gls{PPOS} context, it simply doesn't yield a material advantage. 

\pagebreak

\subsection{Fees}
As described above in \ref{subsec:eco}, serious consideration must be given to the Fee dynamics in the short to medium 
term. Whilst we absolutely need to ensure Algorand stays inexpensive to use, we must also balance this with the need to 
secure the network in perpetuity. This may imply a Fee market, Fee increase or both. Notwithstanding, Algorand's vision 
remains unchanged. A vision which precludes a costly fee model.

\pagebreak

\section{Conclusion}
We believe that incentivised consensus represents the right next step for the protocol. This design should yield a more 
decentralised and secure network, and critically, one which is secure in perpetuity, even in the context of a highly 
uniform token distribution. \\ 

The implementation of the changes discussed herein is a work in progress, the following PRs track that work:


\begin{itemize}
    \item \emph{PR 5740} - \textbf{Incentives: Implements "Mining" - diverting a portion of Fees to proposers}:\\
          \href{https://github.com/algorand/go-algorand/pull/5740}{github.com/algorand/go-algorand/pull/5740} \\
    \item \emph{PR 5757} - \textbf{Incentives: Suspend "absentee" accounts that don't propose}:\\
          \href{https://github.com/algorand/go-algorand/pull/5757}{github.com/algorand/go-algorand/pull/5757} \\
    \item \emph{PR 5799} - \textbf{Incentives: Heartbeat to keep an unlucky node online}:\\
          \href{https://github.com/algorand/go-algorand/pull/5799}{github.com/algorand/go-algorand/pull/5799}
\end{itemize} \mbox{} \\ \\ 


Barring unforeseen issues, we expect production (mainnet) roll-out of consensus incentives by mid 2024.

\pagebreak

\printglossaries

\pagebreak

\end{document}