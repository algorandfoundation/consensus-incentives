\documentclass[11pt,a4paper]{article}

\setlength {\marginparwidth }{2cm}

\usepackage[margin=2.5cm]{geometry}
\usepackage{todonotes}
\usepackage[automake]{glossaries}
\usepackage{graphicx}
\usepackage{microtype}
\usepackage{listings}
\usepackage{color}
\usepackage{amssymb,amsmath}
\usepackage{mathpazo}
\usepackage{accents}
\usepackage{longtable,booktabs}
\usepackage{dcolumn}
\usepackage{pgf}
\usetikzlibrary{shapes.geometric}
\usetikzlibrary{matrix}
\usepackage{natbib}
\usepackage{hyperref}
\usepackage[capitalise,noabbrev,nameinlink]{cleveref}
\hypersetup{
  pdftitle={Algorand Consensus Incentivisation},
  pdfborder={0 0 0},
  breaklinks=true
}
\graphicspath{ {./images/} }

\definecolor{dkgreen}{rgb}{0,0.6,0}
\definecolor{gray}{rgb}{0.5,0.5,0.5}
\definecolor{pink}{rgb}{0.92,0.2,0.86}

\lstset{frame=tb,
  language=C,
  aboveskip=3mm,
  belowskip=3mm,
  showstringspaces=false,
  columns=flexible,
  basicstyle={\ttfamily},
  numbers=none,
  numberstyle=\tiny\color{pink},
  keywordstyle=\color{pink},
  commentstyle=\color{pink},
  stringstyle=\color{pink},
  breaklines=true,
  breakatwhitespace=true,
  tabsize=3
}

\DeclareMathOperator{\dom}{dom}
\newcommand\restrict[2]{\left.#1\right||_{#2}}
\newcommand\deltavar[1]{\accentset{\Delta}{#1}}

\makeglossaries
\newglossaryentry{PPOS}
{
    name={Pure Proof of Stake},
    description={a consensus algorithm based on verifiable random functions over elliptic curves, which provide a 
                 mechanism of cryptographic sortition. Naturally incentivises non-pooled block production.}
}
\newglossaryentry{POW}
{
    name={Proof of Work},
    description={a consensus algorithm based on symmetric hash functions, where a newly mined block's digest is under a 
                 specific threshold. Does not naturally incentivise non-pooled block production.}
}


\begin{document}

\title {Algorand Consensus Incentivisation \\
       {\large \sc An Algorand Foundation discussion paper}}
\date  {Version 0.1, 18th July 2023}
\author{John Woods         \\ {\small \texttt{john@algorand.foundation}} \\
                              {\small \texttt{john@postquantum.dev}} \\
}

\maketitle

\section{Purpose}
The goal of this project is to engineer a native solution at layer 1, modifying the Algorand protocol to 
\emph{incentivise} the execution of consensus via block production rewards in ALGO. \\

\textbf{The technical goals of this project are:}

\begin{enumerate}
    \item Define the economic design (source of funds, payout function, non-functional requirements) - this is the 
          primary Foundation output.
    \item Define the technical design (protocol level augmentations, code-level changes, functional requirements) 
          - secondary goal, Inc will lead.
    \item Define the approach to mitigate gamification (Key surrender, Absenteeism, Vote neglect) - secondary goal.
\end{enumerate}

\pagebreak

\tableofcontents

\pagebreak

\section{Acknowledgements}
The ideas presented herein are inspired by and based on discussions with
people from Algorand Foundation, Algorand Inc, including the research and engineering groups, specifically: Silvio 
Micali, Paul Riegle, Gary Malouf, Eric Wragge, Staci Warden \& Bruno Martins.

\pagebreak

\section{Context}
Algorand employs \gls{PPOS} (glossary below) as a means of reaching consensus. The security of the decentralised network 
is predicated on the total count of Algo tokens which are actively staked at any given moment, as a percentage of the 
total circulating supply. Notably, Algorand's security assumptions rely upon at least $\sim$67\% of the underlying stake 
being honest in order for the network to remain impervious to attack, rather than $\sim$51\% in traditional \gls{POW} 
based systems. \\ 

Algorand's original thesis contended, a priori, that a sufficient number of end-users would be naturally incentivised to 
run a node and contribute to active stake, motivated by the desire to secure their own funds, and as a corollary the 
network. However, a posteriori, it has been observed that as a consequence of the cost to run a participation node with 
high uptime, and indeed the technical expertise required to both instantiate and maintain a node, a critical mass of 
end-user contribution has not been reached. \\

In fact, as the distribution of large holders has slowly fragmented, and a larger proportion of the supply has been held 
by an increasingly less concentrated set of individuals, active stake has notably \emph{dropped}. The aforementioned 
friction is too much for most individuals to contribute actively to network security. \\ \\

This paper proposes potential solutions, at a protocol level, to encourage end-users to actively stake their Algo.
Additionally, we introduce the concept of financial incentives to reward staking, with the aim of materially increasing 
the total percentage of Algo in circulating supply which are participating in consensus. 

\pagebreak

\section{Solution design}

\subsection{Economic considerations}

\subsection{Payout function}
$\mathcal{A}$ 

\subsection{Frequency}

\subsection{Game mitigations}

\subsection{Attacks/Fragility}
\todo{lorem}

\subsection{Fees}

\pagebreak

\section{Conclusion}

\pagebreak

\printglossaries

\end{document}
